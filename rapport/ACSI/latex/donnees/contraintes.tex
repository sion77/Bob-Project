\paragraph{Confidentialité}

Lors de l'inscription, l'internaute n'entre pas ses données personnelles ( Nom, Adresse ... ) afin de ne pas lui demander trop vite
d'informations personelles, ce qui pourrait être perçu comme une intrusion dans sa vie privée alors que ces données ne sont pas encore
nécessaires. Elle lui seront demandées seulement lors de la validation d'une commande ( Utilisation non prévue dans le cadre de cette étude )

\paragraph{Rapidité de Réponse}
	
La grande majorité des images du site seront enregistrées dans un format tel qu'elles seront légères et rapides a charger pour l'internaute.
Le code javascript et les "animations" seront réduites au stricte minimum afin de ne pas ralentir le site.

\paragraph{Charte graphique}

(En annexe)

\paragraph{Normes Ergonomiques}
	
(En annexe)

\paragraph{Portabilité de l'application}
	
Cette application sera conçue de sorte qu'elle soit accessible depuis n'importe quel navigateur récent. Certains éléments graphiques différèrent légèrement selon le navigateur de l'utilisateur ( par exemple une très légère différence sur les couleurs du fond, du a une non-compatibilité de certaines propriétés CSS3 par certains navigateurs) mais cela n'altèrera en rien les fonctionnalités et le confort de navigation.
La résolution d'écran nécessaire au bon affichage du site sera assez faible pour que le plus grand nombre puisse l'utiliser.

\paragraph{Modification du contexte}
	
Le site Brico-Bob sera conçu de telle sorte que ce modèle de magasin en ligne sera transposable à d'autres secteurs ( par exemple de la vente de chocolats ou  une agence de voyages... ) Il suffira de modifier le contenu de la base de données pour répondre a de nouveaux besoins.

\paragraph{Matériel et logiciels utilisés}
	
Le code a été écrit intégralement sur Notepad ++, qui est simple d'utilisation et très clair.
Les tests ont été utilisés grâce à WAMP et nous avons utilisés mysql pour le stockage des données.
Nous avons utilisés GitHub afin de synchroniser notre travail et de nous organiser.
Des tests ont été effectués sur Internet Explorer 7,8,9 Google Chrome et Mozilla Firefox
