1 Le système propose à l'internaute de s'inscrire.
2 L'internaute clique sur "s'inscrire".
3 Le système demande à l'internaute de lui fournir un nom de compte, un mot de passe et une vérification du mot de passe.
4 L'internet entre les informations demandées.
5 L'internaute clique sur "s'inscrire".
6 L'internaute est inscrit et connecter.

Exceptions: 
			5a L'internaute a entrer un pseudo éxistant.
			6 Le systeme affiche "Pseudo déja utiliser".
			7 Le système propose de nouveau à l'internaute de s'incrire.
			
			5a L'internaute a entrer deux mot de passe differents.
			6 Le système affiche "Les mots de passe ne sont pas identiques".
			7 Le système propose de rentrer à nouveau les mots de passe.

\begin{boxedminipage}[t]{12cm}
	\begin{itemize}
		\item Système : Site web "Chez Bob"
		\item Acteur : Internaute
		\item Objectif : Inscrire un compte
		\item Pré-condition : (aucune)
	\end{itemize}

	\renewcommand\theenumi{\arabic{enumi}}
	\renewcommand\labelenumi{\theenumi .}
	\renewcommand\theenumii{\Alph{enumii}}
	\renewcommand\labelenumii{(\theenumii)}
	\paragraph{Scénario :} 
	\begin{enumerate}
		\item \label{sc1l1} un item
	\end{enumerate}
\end{boxedminipage}
\newpage

\begin{boxedminipage}[t]{12cm}
	\begin{itemize}
		\item Système : Site web "Chez Bob"
		\item Acteur : Internaute
		\item Objectif : Se connecter sur le compte de l'utilisateur
		\item Pré-condition : L'utilisateur est enregistré
	\end{itemize}

	\renewcommand\theenumi{\arabic{enumi}}
	\renewcommand\labelenumi{\theenumi .}
	\renewcommand\theenumii{\Alph{enumii}}
	\renewcommand\labelenumii{(\theenumii)}
	\paragraph{Scénario : }
	\begin{enumerate}
		\item \label{sc2l1} Le système propose à l'utilisateur de se connecter à son compte.
		\item \label{sc2l2} L'utilisateur clique sur "se connecter".
		\item \label{sc2l3} Le système demande le nom de compte et le mot de passe de l'utilisateur.
		\item \label{sc2l4} L'internaute entre les informations demandées.
		\item \label{sc2l5} L'internaute clique sur "se connecter".
		\item \label{sc2l6} L'internaute est connecté à son compte.
	\end{enumerate}

	\renewcommand\theenumi{\Alph{enumi}}
	\renewcommand\labelenumi{\theenumi )}
	\renewcommand\theenumii{\arabic{enumii}}
	\renewcommand\labelenumii{\theenumii .}
	\paragraph{Exceptions :} 
	\begin{enumerate}
		\item
		\begin{enumerate}
			\addtocounter{enumii}{4}
			\item Une des informations ou les deux demandées ne correspondent pas.
			\item Le système affiche un message d'erreur
			\item Retour à \ref{sc2l3}
		\end{enumerate}
	\end{enumerate}
\end{boxedminipage}
\newline

\begin{boxedminipage}[t]{12cm}
	\begin{itemize}
		\item Système : Site web "Chez Bob"
		\item Acteur : Internaute
		\item Objectif : Se déconnecter de son compte
		\item Pré-condition : L'utilisateur est connecté à son compte
	\end{itemize}

	\renewcommand\theenumi{\arabic{enumi}}
	\renewcommand\labelenumi{\theenumi .}
	\renewcommand\theenumii{\Alph{enumii}}
	\renewcommand\labelenumii{(\theenumii)}
	\paragraph{Scénario :} 
	\begin{enumerate}
		\item \label{sc3l1} Le système propose à l'utilisateur de se déconnecter de son compte.
		\item \label{sc3l2} L'utilisateur clique sur "se déconnecter".
		\item \label{sc3l3} Le système déconnecte le membre de son compte.
		\item \label{sc3l4} Le système redirige l'internaute vers l'accueil.
	\end{enumerate}
\end{boxedminipage}
\newpage

% On remet les valeurs qui sont là par défaut.
\renewcommand\theenumi{\arabic{enumi}}
\renewcommand\labelenumi{\theenumi .}
\renewcommand\theenumii{\Alph{enumii}}
\renewcommand\labelenumii{(\theenumii)}