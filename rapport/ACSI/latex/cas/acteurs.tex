Il existe 3 acteurs principaux au sein de cette application :

\subsubsection{internaute} 

l'internaute est un visiteur de passage, qui peut venir pour la premiere fois sur le site, pour rechercher et consulter des produits, par exemple pour comparer
les prix avec la concurence. il peut également etre un visiteur regulier qui verifie les promotions en cours. bien que cet internaute puisse accèder a une grande partie des pages du site, ses possibilités sont limités, il a donc la posibilité de s'inscrire afin de devenir membre

\subsubsection{membre}

le membre est un internaute qui s'est inscrit au site, en ne donnant aucune information personnelle, apart son e-mail, necessaire en cas de problème. 
il aura la possibilité de poser des questions sur des produits, de donner son avis ainsi que de noter les articles, mais surtout d'acheter ou de louer des produits.
Le membre ne fournirra ses coordonnées que lors de sa premiere commande, ce pour lui demander les informations necessaires en temps voulu, qu'il ne fournisse pas son nom alors que ce n'est pas encore nesessaire, pour des questions de protection de la vie privée.

\subsubsection{admin}

les administrateurs du site pourront faire tout ce que les membres peuvent faire, mais auront accès a un panneau d'administration, que nous verrons en details plus tard dans ce dossier, qui leur permettra de gerer tout le contenu du site ainsi que les membres, sans passer directement pas le SGBD, ce qui est plus simple d'utilisation (un admin pourra ajouter des produits, promotions, bannir un membre... voir user case  ).
Toutefois, l'administateur ne pourra pas modifier un message ecrit pas un membre, ni les informations sur un membre.
