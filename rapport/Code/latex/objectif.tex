L'objectif de ce projet était de réaliser un site web pour un magasin de bricolage, Brico-Bob.  Ce site devra gérer plusieurs aspects : une partie privée et une partie publique. La partie publique sera accessible a n'importe qui, et comprendra, entre autres, la possibilités de visionner des produits, d'en rechercher, de commenter un produit. La partie privée sera réservée aux administrateurs du site et comprendra un panneau permettant de gérer tout le contenu du site, des produits, images, catégories...  
		\section{Objectifs du site}
		L'objectif de notre site est de proposer à la vente comme à la location des articles de bricolage pour le grand public.  Le site devra donc être assez ergonomique et  attrayant visuellement afin de ne pas repousser les visiteurs.\\
		
Voila les points importants sur lesquels sur lesquels nous avons fait attention :
			\begin{itemize}
\item Un temps de chargement des pages optimisé. En effet On estime le seuil d'acceptation de chargement d'un site à deux secondes (selon plus de 47\%  des internautes sondés).  Au delà, vous commencez à perdre des visiteurs. ( source : Marketing Management par  P. Kotler ).\\

\item Proposer une description des produits assez détaillée : si le client ne trouve 	pas 	assez d'informations sur un produit sur notre site, il ira chercher ces 	informations chez un concurrent, et a moins que notre prix soit 	significativement plus bas, il y a de grandes chances qu'il achète chez ce 	concurrent.\\

	
\item Il faudra utiliser le plus de termes simples a comprendre par l'utilisateur plutôt 	que des termes techniques barbares.  Notre site permet donc d'écrire de 	longues description, mais cette responsabilité incombe plutôt à la société Brico-	Bob.\\

\includegraphics[scale=0.5]{demofichepro.jpg}

\item Donner des informations sur la société et sur le service client, ainsi que les 	administrateurs du site. Le client veut savoir a qui il a affaire et sera plus enclin 	à acheter si il sait qu'il peut contacter quelqu'un ensuite. Le client sera mis en 	confiance, ce qui est vraiment important lorsque son argent est en jeu.\\
	Ces informations seront donc accessibles depuis n'importe quelle page du site 	grâce au menu :\\

\includegraphics[scale=0.5]{menu.jpg}

\item Ne pas trop s'introduire dans la vie privée du client : nous n'avons demandé 	au client que des informations non personnelles lors de son inscription :  son 	pseudo et un mot de passe. Il n'as pas besoin de donner son nom ni son 	adresse tant qu'il ne valide pas une commande. Cela lui permettra de ne pas se  	sentir "tracké" et de visiter librement le site.\\

\item Un moteur de recherche bien réalisé. Si le client sait exactement ce qu'il veut 	lorsqu'il va sur le site, il ira chercher le nom du produit qu'il a en tête dans la 	barre de recherche. Si le moteur de recherche n'est pas bien conçu, il risque de 	penser que nous n'avons pas le produit qu'il cherche et il ira donc voir ailleurs. 	Si il n'a qu'une idée très vague, il y a alors des chances qu'il se tourne vers le 	moteur de recherche avancée. Cela nous permettra de cerner ses critères de 	prix par exemple.\\

\item Mettre en évidence le produit en mettant une image assez grande : trop de site proposent encore des images ridiculement petites, ce qui peut décourager le client d'acheter, car il ne peut pas bien voir le produit et peut même penser qu'il y a des risques de "tromperie" sur la marchandise.\\

\item Une navigation aisée, c'est à dire des catégories bien organisées. Bien que notre site permette d'afficher à la fois des produits et des catégories sur une même page, cela sera à éviter dans le cas général. Enfin, il faudra veiller à ne pas créer de catégories vides, qui prendrait de la place pour rien et serait une source de  frustration pour le client ( quoi de plus désagréable que de trouver une catégorie qui correspond parfaitement à ce que l'on recherche et se rendre compte qu'elle est vide )\\

\item Mettre l'accent sur les produits : le but d'un site de e commerce est de vendre, et nous avons donc fait attention de mettre l'accent sur les produits. Le design est assez sobre, et les autres éléments du site 	n'empiètent as sur les produits. Il y a également sur la page d'accueil des "coups de cœur" permettant de stimuler un achat imprévu chez le client.

\includegraphics[scale=0.5]{coupcoeur.jpg}


\end{itemize}