\subsection{header.php}
				Ce fichier va inserer la bibliothèque Smarty et les déclarations de classes, puis il va déclarer des constantes et crée quelques fonctions  
			\subsection{index.php}
				Ce fichier va tout d'abord appeller header.php puis va creer l'objet Bob et smarty.\\
				Il va ensuite analyser la requete pour savoir quel template il va appeller pour fabriquer la page et va effectuer les actions qu'on lui demande de faire.\\
				Il va ensuite passer à smarty plusieurs variables puis generer la page.
			\subsection{les fichiers de /classes}
				Ces fichiers contiennent les classes qui contiennent les infos de la base de données et qui la manipule.
			\subsection{les fichiers de /templates}
				Ils permettent d'afficher une page en utilisant les variables qu'on lui prete, il y a deux style de templates : 
				\begin{itemize}
				\item les fichiers modèle : ils se situent dans /templates/modele
				\begin{itemize}
					\item main.tpl : Contient ce que toute page va contenir, il va appeller les autres fichiers templates de modèle
					\item menu.tpl : Contient le menu du site
					\item entete.tpl  : Contient les informations sur la page ($<head>$)
					\item espace\_membre.tpl : Contient l'espace membre\\
				\end{itemize}
					
				\item le fichier appelé pour la page : ils se situent dans /templates.\\
				Il y en a un par page
				\end{itemize}
			\subsection{les fichiers de /css}
				Ils permettent de coder le design du site, ils se situent dans /css/. Ces fichiers sont séparés en plusieurs fichiers. On préférera en mettre un général puis un par page.
			
			