% Le code
\lstset{language=PHP}
\begin{lstlisting}
<?php
// (...)
// Nous sommes dans Bob
private function initMembres()
{
    // Au depart : pas de membres.
    $this->nbMembres = 0;
    
    // Rappel : Admin est une classe speciale de Membre

    // On fait notre requete qui a pour but de recuperer 
    // les membres et administrateurs
    $req = $this->query("  
\end{lstlisting}
\lstset{language=SQL}
\begin{lstlisting}
    SELECT idUtilisateur AS \"id\",
        pseudoUtilisateur AS \"pseudo\",
        passUtilisateur AS \"pass\",
        '0' AS \"admin\"
    FROM utilisateur
    WHERE idUtilisateur NOT IN( SELECT idAdmin FROM admin )
                            
UNION
                            
    SELECT idUtilisateur AS \"id\",
           pseudoUtilisateur AS \"pseudo\",
           passUtilisateur AS \"pass\",
           '1' AS \"admin\"
    FROM utilisateur
    WHERE idUtilisateur IN( SELECT idAdmin FROM admin )
                            
ORDER BY id"
\end{lstlisting}
\lstset{language=PHP}
\begin{lstlisting}
                        );            

    // Pour chaque ligne du resultat
    while($rep = $req->fetch())
    {
        // On stocke le membre dans le bon objet
        if($rep["admin"])
            $m = new Admin($this, $rep["id"], 
                           $rep["pseudo"], $rep["pass"]);
        else
            $m = new Membre($this, $rep["id"], 
                            $rep["pseudo"], $rep["pass"]);
            
        // On ajoute le membre au tableau de membres de l'objet Bob
        // (on aurait pu utiliser $this->ajouterMembre($m); )
        $this->membres[$this->nbMembres] = $m;            
        $this->nbMembres++;
    }

    // On libere le resultat
    $req->closeCursor();
    
    return true;
}
// (...)
?>
\end{lstlisting}