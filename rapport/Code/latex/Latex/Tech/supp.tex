% Support du code
Pour le projet nous avons utilisé plusieurs supports :\\
\begin{description}
	\item[Github :] Il s'agit d'un système de partage de projets gratuit -si on accepte de les mettre open-source- très pratique.
	\item[Notepad++ :] Il s'agit de notre éditeur de texte favori, nous l'avons tous utilisé
	\item[Wamp/Lampp :] Il s'agit d'un émulateur de serveur Apache/MySql/Php pour pouvoir tester le site en local.
		\begin{description}
			\item[Apache :] Il émule le serveur en lui meme, il contient notemment une interface "phpMyAdmin" et "Wamp" (pour les différents projets)
			\item[MySql :] Il émule la base de données
			\item[Php :] Le programme qui, suivant la requete, interpretera la bonne page php.
		\end{description}
	\item[The GIMP :] C'est un outil pour manipuler les images très puissant : beaucoup d'éléments ont été réalisées grace à lui.
\end{description}