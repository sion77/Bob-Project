% Langages

Nous avons utilisé, comme pour la plupart des sites, les langages :
\begin{description}
	\item[HTML :] Il défini les éléments de la page dans les templates	
	\item[CSS :] Il gère le design du site, nous avons séparé ses fichiers en plusieurs parties (Au départ il s'incluait tous, mais nous avons fait en sorte que Smarty n'appelle que ceux qui nous importe)
	\item[PHP :] Il permet de creer la page HTML suivant la requête qu'on lui donne.	
	\item[SQL :] Il est utilisé par Php pour manipuler la base de données, laquelle contient tous les éléments du site, lesquels sont stockés au chargement de la page dans des objets (Php)	
	\item[JavaScript :] Il permet de faire "Bouger la page" chez le client, sans avoir besoin d'envoyer de requête Php.\\
	Cela permet, par exemple, de faire un premier contrôle local des données que le client veut envoyer mais cela permet aussi, de rendre certains elements plus ergonomiques, par exemple pour la notation des produits.
\end{description}