Nous avons utilisé, comme pour la plupart des sites, les langages :
\begin{description}
	\item[HTML :] Utilisé pour la structure des pages web, elle permet de définir les éléments que contiens la page. On s'en est notamment servi pour organiser la page grâce aux balises <div> et <span>.
	
	\item[CSS :] Nous avons géré tout le design de notre site grâce a divers fichiers css, nous n'avons pas inséré de CSS directement dans le HTML.
Le design a été découpe en divers fichiers, chacun étant utile pour le design d'une partie du site. L' "assemblage" des fichiers .css nécessaires
se fait en partie grâce à SMARTY.
	
	\item[PHP :] Utilisé pour rendre le site interactif en iteragissant avec la base de données (avec le langage SQL) et en traitant les données.
	Php va ensuite appeler Smarty et lui passer des données pour afficher une page.\\
	Nous utilisons Php, notemment pour :
	\begin{itemize}
		\item Créer des comptes utilisateurs
		\item Gérer les sessions
		\item Utiliser un panneau administrateur afin de gérer le contenu du site
	\end{itemize}
	
	\item[SQL :] Toutes les catégories, membres, produits, avis sont stockées dans la base de donnée. Nous avons utilisé cette base de donnée pour stocker certaines images telles que les illustrations des produits mis à disposition. Ces installations ont nécessité un certains nombre de contrôle de sécurité mais 	ce sont elles qui permettent d'avoir un site fonctionnel.\\
	Le SQL permet de manipuler la base de données.\\
	On stockera les informations de la base dans les classes Php
	
	\item[JavaScript :] Le JavaScript sur notre site nous sert principalement à contrôler les entrées de l'utilisateur, par exemple sur la page d'inscription afin de contrôler que les informations entrées sont valides.
Nous l'utilisons donc dans certains formulaires du site afin de ne pas provoquer de trop grosse frustration chez l'utilisateur au cas ou il clique sur "valider"
en ayant entré des informations non valides.
Voila un autre exemple d'utilisation du JavaScript, qui permet de rentre le système de notation plus attractif : l'utilisateur note un produit en cliquant sur le
nombre d'étoiles qu'il souhaite lui attribuer.

\end{description}